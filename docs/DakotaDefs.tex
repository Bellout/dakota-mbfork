% Shortcuts for Dakota information incorporated in multiple versions of manuals
%
% This file is included in all top-level .tex files

% When using this command, likely follow with a forced space, e.g., one of
% Version \DakotaVersion\ Manual
% Version \DakotaVersion\space Manual
% Version \DakotaVersion{} Manual
\newcommand{\DakotaVersion}{6.3}


% Most recent SAND report numbers and report date based on
% Data should be SAND report date and not reflect updates
\newcommand{\DakotaSANDDate}{July 2014}
\newcommand{\DakotaSANDDev}{SAND2014-5014}
\newcommand{\DakotaSANDRef}{SAND2014-5015}
\newcommand{\DakotaSANDUsers}{SAND2014-4633}
\newcommand{\DakotaSANDTheory}{SAND2014-4253}


% Need to manually break line or Eldred gets incorrectly split
\newcommand{\DakotaAuthorSAND}{
Brian~M.~Adams, Lara~E.~Bauman, William~J.~Bohnhoff, Keith~R.~Dalbey,\\ 
John~P.~Eddy, Mohamed~S.~Ebeida, Michael~S.~Eldred, Russell~W.~Hooper,\\
Patricia~D.~Hough, Kenneth~T.~Hu, John~D.~Jakeman, Ahmad~Rushdi,\\
Laura~P.~Swiler, J.~Adam~Stephens, Dena~M.~Vigil, Timothy~M.~Wildey
}

\newcommand{\authvskip}{0.25cm}

\newcommand{\DakotaAuthorLong}{
{Brian~M.~Adams, Mohamed~S.~Ebeida, Michael~S.~Eldred, John~D.~Jakeman,}\\
{Laura~P.~Swiler, J.~Adam~Stephens, Dena~M.~Vigil, Timothy~M.~Wildey}\\
{\sl Optimization and Uncertainty Quantification Department} \vspace*{\authvskip} \\ 
{William~J.~Bohnhoff,} {\sl Radiation Transport Department} \vspace*{\authvskip} \\
{John~P.~Eddy,} {\sl System Readiness and Sustainment Technologies Department} \vspace*{\authvskip} \\
{Russell~W.~Hooper,} {\sl Multiphysics Applications Department} \vspace*{\authvskip} \\
{Kenneth~T.~Hu,} {\sl Validation and Uncertainty Quantification Department} \vspace*{\authvskip} \\
{Keith~R.~Dalbey,} {\sl Mission Analysis and Simulation Department} \vspace*{\authvskip} \\
{Lara~E.~Bauman, Patricia~D.~Hough} \\
{\sl Quantitative Modeling and Analysis Department} \vspace*{\authvskip} \\
Sandia National Laboratories\\
P.O. Box 5800\\
Albuquerque, NM 87185 \vspace*{\authvskip} \\
{Ahmad~Rushdi,} {\sl Institute for Computational and Engineering Sciences} \vspace*{\authvskip} \\
The University of Texas at Austin\\
POB 4.102
Austin, TX 78712\\
}

\newcommand{\DakotaAuthorFormatted}{
{\large \bf Brian M. Adams, Mohamed S. Ebeida, Michael S. Eldred, John D. Jakeman,\\
Laura P. Swiler, J. Adam Stephens, Dena M. Vigil, Timothy M. Wildey}\\ 
{\large Optimization and Uncertainty Quantification Department}\\
\vspace*{\authvskip}
{\large \bf William J. Bohnhoff}\\
{\large Radiation Transport Department}\\
\vspace*{\authvskip}
{\large \bf Keith R. Dalbey}\\
{\large Mission Analysis and Simulation Department}\\
\vspace*{\authvskip}
{\large \bf John P. Eddy}\\
{\large System Readiness and Sustainment Technologies Department}\\
\vspace*{\authvskip}
{\large \bf Russell W. Hooper}\\
{\large Multiphysics Applications Department}\\
\vspace*{\authvskip}
{\large \bf Kenneth T. Hu}\\
{\large Validation and Uncertainty Quantification Department}\\
\vspace*{\authvskip}
{\large \bf Lara E. Bauman, Patricia D. Hough}\\
{\large Quantitative Modeling and Analysis Department}\\
\vspace*{\authvskip}
{\large Sandia National Laboratories}\\
{\large P.O. Box 5800}\\
{\large Albuquerque, New Mexico 87185}\\
\vspace*{\authvskip}
{\large \bf Ahmad Rushdi}\\
{\large Institute for Computational and Engineering Sciences}\\
\vspace*{\authvskip}
{\large The University of Texas at Austin}\\
{\large POB 4.102}\\
{\large Austin, TX 78712}\\
}


% Fragments which together comprise abstracts

\newcommand{\DakotaAbstractShared}{
The Dakota (Design Analysis Kit for Optimization and Terascale
Applications) toolkit provides a flexible and extensible interface
between simulation codes and iterative analysis methods. Dakota 
contains algorithms for optimization with gradient and
nongradient-based methods; uncertainty quantification with sampling,
reliability, and stochastic expansion methods; parameter
estimation with nonlinear least squares methods; and
sensitivity/variance analysis with design of experiments and parameter
study methods. These capabilities may be used on their own or as
components within advanced strategies such as surrogate-based
optimization, mixed integer nonlinear programming, or optimization
under uncertainty. By employing object-oriented design to implement
abstractions of the key components required for iterative systems
analyses, the Dakota toolkit provides a flexible and extensible
problem-solving environment for design and performance analysis of
computational models on high performance computers.
% blank line intended

% blank line intended
}

\newcommand{\DakotaAbstractDev}{
This report describes the Dakota class hierarchies. It is derived 
from annotation of the source code and provides detailed class 
documentation, including all member functions and attributes.  
}

\newcommand{\DakotaAbstractRef}{
This report serves as a reference manual for the commands specification
for the Dakota software, providing input overviews, option descriptions,
and example specifications.
}

\newcommand{\DakotaAbstractUsers}{
This report serves as a user's manual for the Dakota software and
provides capability overviews and procedures for software execution,
as well as a variety of example studies.
}

\newcommand{\DakotaAbstractTheory}{
This report serves as a theoretical manual for selected algorithms
implemented within the Dakota software.  It is not intended as a
comprehensive theoretical treatment, since a number of existing texts
cover general optimization theory, statistical analysis, and other
introductory topics.  Rather, this manual is intended to summarize a
set of Dakota-related research publications in the areas of
surrogate-based optimization, uncertainty quantification, and
optimization under uncertainty that provide the foundation for many 
of Dakota's iterative analysis capabilities.
}
