\chapter*{Preface}
\addcontentsline{toc}{chapter}{Preface}
%\chapter{Preface}

The DAKOTA (Design Analysis Kit for Optimization and Terascale
Applications) project started in 1994 as an internal research and
development activity at Sandia National Laboratories in Albuquerque,
New Mexico. The original goal of this effort was to provide a common
set of optimization tools for a group of engineers who were solving
structural analysis and design problems. Prior to the start of the
DAKOTA project, there was not a focused effort to archive the
optimization methods for reuse on other projects. Thus, for each new
project the engineers found themselves custom building new interfaces
between the engineering analysis software and the optimization
software. This was a particular burden when attempts were made to use
parallel computing resources, where each project required the
development of a unique master program that coordinated concurrent
simulations on a network of workstations or a parallel computer. The
initial DAKOTA toolkit provided the engineering and analysis community
at Sandia Labs with access to a variety of different optimization
methods and algorithms, with much of the complexity of the
optimization software interfaces hidden from the user. Thus, the
engineers were easily able to switch between optimization software
packages simply by changing a few lines in the DAKOTA input file. In
addition to applications in structural analysis, DAKOTA has been
applied to applications in computational fluid dynamics, nonlinear
dynamics, shock physics, heat transfer, and many others.

DAKOTA has grown significantly beyond its original focus as a toolkit
of optimization methods. In addition to having many state-of-the-art
optimization methods, DAKOTA now includes methods for global
sensitivity and variance analysis, parameter estimation, and
uncertainty quantification, as well as meta-level strategies for
surrogate-based optimization, mixed-integer nonlinear programming,
hybrid optimization, and optimization under uncertainty. Underlying
all of these algorithms is support for parallel computation; ranging
from the level of a desktop multiprocessor computer up to massively
parallel computers found at national laboratories and supercomputer
centers.

This document corresponds to DAKOTA Version 5.0.  Release notes for
this release, past releases, and current developmental releases are
available from
\url{http://www.cs.sandia.gov/dakota/licensing/release_notes.html}.

As of Version 5.0, DAKOTA is publicly released as open source under a
GNU Lesser General Public License and is available for free download
world-wide.  See \url{http://www.gnu.org/licenses/lgpl.html} for more
information on the LGPL software use agreement.  DAKOTA Versions 3.0
through 4.2+ were licensed under the GNU General Public License.  The
objective of DAKOTA public release is to facilitate research and
software collaborations among the developers of DAKOTA at Sandia
National Laboratories and other institutions, including academic,
governmental, and corporate entities. For more information on the
objectives of the open source release and how to contribute, refer to
the DAKOTA FAQ at \url{http://www.cs.sandia.gov/dakota/faq.html}.

The DAKOTA leadership team consists of Brian Adams (project lead),
Mike Eldred (research lead), Bill Bohnhoff (acting support manager),
and Jim Stewart (business manager).  DAKOTA development team members
include Keith Dalbey, John Eddy, David Gay, Karen Haskell, Patty
Hough, and Laura Swiler.  Additional historical contributors to DAKOTA
and it's third-party libraries are acknowledged on the DAKOTA web
page.

\textbf{Contact Information}:

{\small Brian M. Adams, DAKOTA Project Lead}\\
{\small Sandia National Laboratories}\\
{\small P.O. Box 5800, Mail Stop 1318}\\
{\small Albuquerque, NM 87185-1318}

{\small {\bf User community/help:} \href{mailto:dakota-users@software.sandia.gov}{dakota-users@software.sandia.gov}}\\
{\small {\bf Development team:} \href{mailto:dakota-developers@development.sandia.gov}{dakota-developers@development.sandia.gov}}\\
{\small {\bf Web:} \url{http://www.cs.sandia.gov/dakota}}
