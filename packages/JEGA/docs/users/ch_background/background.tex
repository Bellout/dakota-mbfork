\chapter{Background} \label{ch:background}

JEGA is the evolution of work began in 1999 at the University at
Buffalo, SUNY in Buffalo, NY while persuing a masters degree.  At
the time the package had no official name and contained only a
multi-objective genetic algorithm.  The operations are detailed
in~\cite{eddy:ms:2001}.

Over the next two years after completion of the masters thesis, the
algorithm was refined and re-factored and built into a larger
package as an optimization component.  This work was also completed
at SUNY Buffalo by John Eddy.  This package was never made publicly
available. It included a handful of optimization methods and a
custom visualization technique called Cloud
Visualization~\cite{eddy:cloud_visualization:2002}.

In the summer of 2003 while still working on a Ph.D. John was hired
on at Sandia labs as a technical intern working in the Optimization
and Uncertainty Estimation department.  During the three month
internship, the code was again re-factored and made a sub-package of
\href{http://dakota.sandia.gov}{The DAKOTA Project}
and was given the name JEGA. It was also at this time that a
separate single objective GA was added to the package.

With the exception of minor bug fixes, the code available through
DAKOTA remained the same until December 2006. During the interim
time, John continued to modify, extend, and improve JEGA in support
of his doctoral work.  Once finished with his doctorate, John was
hired on as a permanent member of the technical staff at Sandia in
the System Readiness and Sustainment department.  In the time since,
JEGA has undergone a great deal of development but the software
design has remained conceptually in tact.  Many additional
capabilities, bug fixes, new operator types, new operator
specializations, etc. have been added.

In September 2006, JEGA was officially migrated out of the DAKOTA
project and became an independent Sandia software development
project.  That is the current status of JEGA.
